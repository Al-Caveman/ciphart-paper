\documentclass{article}
\usepackage{amsmath}
\usepackage{algorithmic}

\author{caveman}
\title{a sequential memory-hard key derivation function with better
measurable security}
\begin{document}
\maketitle
\begin{abstract}
hi --- i propose \emph{ciphart}, a sequential memory-hard key derivation
function that has a security gain that's measurable more objectively and
more conveniently than anything in class known to date.

to nail this goal, \emph{ciphart}'s security gain is measured in the unit
of \emph{relative entropy bits}.  relative to what?  relative to the
encryption algorithm that's used later on.  therefore, this \emph{relative
entropy bits} measure is guaranteed to be true when the encryption
algorithm that's used with \emph{ciphart} is also the same one that's used
to encrypt the data afterwards.
\end{abstract}

\section{intro}
first i'll describe the ciphart algorithm, then i will tell you why it's
memory hard, and how it offers better measurable security.

\section{ciphart}

\begin{tabular}{lll}
    \textbf{input:}  & &\\
             & $e$ & number of entropy bits to be added.\\
             & $k$ & initial key.\\
             & $f$ & encryption function.\\
             & $m_i$ & memory pad.\\
    \textbf{output:} & &\\
             & $\hat k$ & better key.\\
    \textbf{steps:}  & &\\
\end{tabular}

\begin{algorithmic}
    \STATE define $p,t,r$ such that $ptr - 2^e$ is smallest positive
    number.
    \FOR{$p=0$ to $p=P$}
        \FOR{$t=0$ to $t=T$}
            \FOR{$r=0$ to $r=R$}
                \STATE $n \leftarrow p \oplus t \oplus r$
            \ENDFOR
        \ENDFOR
    \ENDFOR
\end{algorithmic}

\end{document}
