\documentclass[twocolumn]{article}
\usepackage{amsmath}
\usepackage{algorithmic}

\author{caveman}
\title{sequential memory-hard key derivation \\
with better measurable security}
\begin{document}
\maketitle
\begin{abstract}
hi --- i propose \emph{ciphart}, a sequential memory-hard key derivation
function that has a security gain that's measurable more objectively and
more conveniently than anything in class known to date.

to nail this goal, \emph{ciphart}'s security gain is measured in the unit
of \emph{relative entropy bits}.  relative to what?  relative to the
encryption algorithm that's used later on.  therefore, this \emph{relative
entropy bits} measure is guaranteed to be true when the encryption
algorithm that's used with \emph{ciphart} is also the same one that's used
to encrypt the data afterwards.
\end{abstract}

\tableofcontents

\section{intro}
first i'll describe the ciphart algorithm, then i will tell you why it's
memory hard, and how it offers better measurable security.

\section{ciphart}

\begin{tabular}{lll}
    \textbf{input:}  & $b$ & number of entropy bits to be added.\\
            & $k$ & initial key.\\
            & $f$ & encryption function.\\
            & $m_i$ & memory pad, at least $32$ bytes.\\
            & $R$ & number of rounds per task.\\
    \textbf{output:} & $\hat k$ & better key.\\
\end{tabular}
\begin{algorithmic}[1]
    \STATE define $P,T$ such that $PTR - 2^b$ is smallest positive number,
    and that $T$ is an even number.
    \FOR{$p=1$ to $P$}
        \FOR{$t=1$ to $T$ in steps of $2$}
            \STATE $a \leftarrow t$
            \STATE $b \leftarrow t+1$
            \FOR{$r=1$ to $2R$}
                \STATE $n \leftarrow p ^\frown t ^\frown r$
                \STATE $m_a \leftarrow f(m_b, k, n)$
                \STATE $\hat a \leftarrow a$
                \STATE $a \leftarrow b$
                \STATE $b \leftarrow \hat a$
            \ENDFOR
        \ENDFOR
    \ENDFOR
\end{algorithmic}

\section{sequential-memory hardness}
\section{better security interpretation}

\end{document}
